\chapter{Computer Systems: Complex Behavior from Simple Principles}

Computer systems, both hardware and software, are some of the
most complicated artifacts that humans have ever created. But at
their very core, computer systems are straightforward
applications of the same logical principles that philosophers
have been developing for more than two thousand years, and
this book will show you how.

The hardware component of a computer system is made up of the visible
pieces of equipment that you are familiar with, such as monitors,
keyboards, printers, web cameras, and USB storage devices.  It also
includes the 
components inside the computer, such as chips, cables, hard
drives, DVD drives, and boards. 
The properties of hardware devices are largely fixed when the system is
constructed. For example, a hard drive can store 2 gigabytes,
or a cable may be able to transmit 25 different signals
simultaneously---but no more. 

The hardware that makes up a computer system is not much different
than the electronics inside a television set or DVD
player. But computer hardware can do something that other
consumer electronic devices cannot; it can respond to 
information encoded in a special way, what is 
referred to as the software component of the system.  We usually
call the software a ``program.''

You may have heard
about different computer chips and their instruction sets, such
as the Intel chips that power the laptop we are using to
write this book.  Software for these chips consists of a list of
instructions, and that is the way that most general-purpose computer
systems in use today represent software.
But there is nothing magical about this way of thinking about
software; in fact, we believe that thinking about software in this 
way obscures its important features.  For example, during the 1980s
and 1990s, special computers were created that used mathematical
functions as the basis for the software.  And other computer
systems have completely different representations for software.
The Internet service Second Life, for example, can be programmed 
using Scratch for Second Life (S4SL), and programs in S4SL look
like drawings, not instructions.  An even more graphical view of
software is provided by LabView, which is used by scientists and
engineers all over the world to control laboratory equipment.
If you are at a university, there is a good chance
that LabView is being used at one of the laboratories near you!
And software in LabView looks like a large engineering drawing.
In this book, we think of software as consisting of mathematical
formulas.  This view is entirely consistent with the view of
software as a list of instructions or a specialized drawing.  More
important, it makes computer software accessible to anyone with a 
knowledge of high school algebra.

\begin{aside}
Logicians and mathematicians have been studying models of computation
since before computers were invented. This was done, in part, to
answer a deep mathematical question: What parts of mathematics can,
in principle, be fully automated?  In particular, is it possible to
build a machine that can discover all mathematical truths?

As a result, many different models of computation were developed,
including Turing machines, Lambda calculus, partial recursive
functions, unrestricted grammars, Post production rules, ramdom-access
machines, and many others.  Historically, computer science theory has
treated Turing machines as the canonical foundation for computation,
although modern computers are more closely described by the
random-access model.  Our equational model of computation is closely
related to the Lambda calculus.

What is truly remarkable is that all of these different models of 
computation have been shown to be equivalent.  That is,
if a computation can be described in one of these models, it can also
be described in any of the other models.  It is natural, therefore to
extend this observation by conjecturing that all ``reasonable'' models
of computation are equivalent.  This conjecture is called the
Church-Turing Thesis, and it lies at the heart of computer science
theory.

Also remarkable is that some problems cannot be solved by a program
written in any of these computational models.  Of course, since the
models of computation are equivalent, if a problem cannot by solved by
one of the models, it cannot be solved in any of the models.  Alan
Turing, who many consider to be the first theoretical computer
scientist, was the first to discover such uncomputable problems,
shortly after the logician Kurt G\"odel showed that no formal system
of logic could prove all mathematical truths.  Turing's and G\"odel's 
discoveries of uncomputability and incompleteness show that it is
not possible to build a machine that can discover all mathematical
truths, not even in principle.

\caption{Models of Computation}
\label{aside-model-of-computation}
\end{aside}

It is the software that gives a computer system much of its power and
flexibility. An iPhone, for example, has a screen that
can display 441,600 different pixels arranged in a 960x460
grid, but it is the software that determines whether the
iPhone displays an album cover or a weather update. 
Software makes the hardware more useful by extending the way
it behaves. For instance, a speaker may only be able to produce a
single tone at a time. But the software can instruct the speaker to
play a sequence of tones that sounds like Beethoven's Fifth Symphony.

The mental image you should have is that hardware consists of
the parts in a computer system that you can see, and software
consists of the information that tells the system what to do.
But the distinction between hardware and software is not as
clear cut as this suggests. Many hardware components actually
encode software directly and control other pieces of the system. 
In fact, hardware today is designed and built using many techniques
first developed to build large software projects.  And we will see in
this book that both hardware and software can be modeled using formal
logic.

The distinction between hardware and software is well and good,
but it leaves many questions to the imagination.
At the very least, we must answer the following questions:
\begin{enumerate}
\item How can software affect hardware, e.g., instruct a
        speaker to play a note?
\item How can software detect the hardware’s current status, e.g.,
        whether a switch is pressed or not?
\item What other instructions can the software give the hardware,
        e.g., add two numbers or move data from one location to
        another?
\end{enumerate}

We'll start with the last question.  The answer to this is provided by
a model of computation, as described in Aside~\ref{aside-model-of-computation}.
In fact, there are as many answers to this question as there are
models of computation, and logicians have been very creative when it
comes to constructing such models!  Luckily, these models are
completely equivalent to each other, so we can choose the answer
that is most convenient to us.  And the most convenient answer is 
that a program consists of basic mathematical primitives, such as
the arithmetic functions, and the ability to define new functions.

Once we take functions as the basic model of computation, then the
answers to the other questions are natural.  The software can affect
the hardware by the values that the function returns.  For example,
a program---that is, a function---can tell an iPhone what to display
in its screen by returning a 960x460 matrix.  Each entry in the matrix
can be a number that represents a color, for example 16,711,680 for
red or 65,280 for green.  Similarly, the hardware can inform the
software of its current status via the function's parameters.  For
example, the input to the function can be the coordinates of the pixel
that the user selected by tapping the screen with her finger.  Of
course, the user may perform different gestures, such as tapping or
scrolling.  Each gesture will trigger a different function.

To make this clear, let's consider a simple example.  We will build a
simple computer device that plays rock-paper-scissors.  The machine
has three buttons, allowing the user to select rock, paper, or
scissors.  The device also has a simple display unit.  After the user 
makes a selection, the display unit shows the computer's choice and
lets the user know who won that round.  Our program will make the
selection and determine the winner.  To keep things fair, we will
write this program as two separate functions, so the selection
function cannot ``see'' the user's choice.

The first function, called \texttt{emily}\footnote{This function is
named after one of the authors' children.  The ``real'' Emily
plays rock-paper-scissors game just like the program developed in this
chapter.}, makes the computer's choice.  The output of this function
is obvious: It should be one of ``rock'', ``paper'', or ``scissors''.
If it makes you feel better, you can think of this output as 0, 1, or 2.
But one of the things that makes computers so useful is that they can work
with any type of information, not just numbers.

But what about the input to the function \texttt{emily}?  Remember
that the value of a mathematical function is completely determined by
its inputs.  That's what it means to be a function! So if
\texttt{emily} has no inputs, it must always return the same value,
and that would be a very boring game.  We've already decided that it
would be unfair for \texttt{emily} to see the user's current choice,
but what about the last round?  We can make our selection by
considering the last round of the game, so the input can be the user's
\emph{previous} choice---or a special token, like ``N/A'' for the
game's first round.  The function \texttt{emily} can now be described
as follows:
\begin{displaymath}
emily(u) =
   \left\{
        \begin{array}{ll}
        \mbox{``rock''}     & \mbox{if } u = \mbox{``scissors''} \\
        \mbox{``paper''}    & \mbox{if } u = \mbox{``rock''} \\
        \mbox{``scissors''} & \mbox{otherwise}
        \end{array}     
   \right.
\end{displaymath}       

The second function, which can be called \texttt{score}, decides who
was the winner of the round.  Its input corresponds to the choices
made by the computer and the user, respectively.  Its output consists
of a pair or values.  The first value determines the winner, and the
second ``remembers'' the user's choice.
\begin{displaymath}
score(c,u) =
   \left\{
        \begin{array}{ll}
        (\mbox{``none''}, u)     & \mbox{if } c = u \\
        (\mbox{``computer''}, u) & \mbox{if } (c,u) = (\mbox{``rock''}, \mbox{``scissors''}) \\
        (\mbox{``user''}, u)     & \mbox{if } (c,u) = (\mbox{``rock''}, \mbox{``paper''}) \\
        (\mbox{``computer''}, u) & \mbox{if } (c,u) = (\mbox{``paper''}, \mbox{``rock''}) \\
        (\mbox{``user''}, u)     & \mbox{if } (c,u) = (\mbox{``paper''}, \mbox{``scissors''}) \\
        (\mbox{``computer''}, u) & \mbox{if } (c,u) = (\mbox{``scissors''}, \mbox{``paper''}) \\
        (\mbox{``user''}, u)     & \mbox{if } (c,u) = (\mbox{``scissors''}, \mbox{``rock''})
        \end{array}     
   \right.
\end{displaymath}       
What is the point of the second value of \texttt{score}?  As you can
see, it is always equal to $u$, i.e., the user's selection.  The
intent is that this second returned value will be passed as the input
to the next call of \texttt{emily}.  This is how the program can
remember the user's previous choice.

We could certainly have dropped this second return value, and simply
stipulated that the hardware is responsible for remembering
the user's last selection.  However, we chose to write the program
in this way for two reasons.  First, it gives us more flexibility.
The hardware is simply responsible for storing the value returned 
by \texttt{score} and sending it to \texttt{emily} in the next round.
We can make the program a more sophisticated player of
rock-paper-scissors by changing the software, possibly changing the
value that is passed from \texttt{score} to \texttt{emily} in the
process.  For example, a more sophisticated program may want to
keep track of the number of times that the user has selected each of
the choices.  Since it's the software that decides what to remember
from each round, this choice can be changed very easily.  That is
the flexibility and power of software.

The second reason for writing the program in this way is that it
gives us an opportunity to make an important point about our
computational model.  Since our programs are written as mathematical
functions, they cannot ``remember'' anything.  Programmers who are
used to other computational models (such as C++ or Java) may be
very surprised by this!  It is natural in those models to store
values in ``variables'' and use those variables later in the program,
but that is not possible in our purely functional model.  However, our
model allows the \emph{hardware} to keep certain values, in much the
same way that the hardware can tell which button was pressed by the
user.  In fact, each button ``remembers'' whether it's pressed or not.
Our model of computation simply allows the hardware to have some
``hidden'' buttons that can store information and later pass it to
another function.

This model is very simple, and you could be forgiven for thinking that
there has to more to it, that computers must be much more complicated.
But in fact, that's all there is.  Consider \textit{Deep Blue}, the
computer that beat world champion Gary Kasparov at chess on May 11,
1997.  This program can be written as a function with a single input,
an 8x8 matrix of numbers that represent the position of the pieces
on the board after Kasparov's last move.  The function's output is 
either the position of the board after its move, or a special
white-flag token used to ``resign'' the game.

In principle, a chess-playing function can be written as follows.
Given the input board, determine all possible legal moves.  If there
are no legal moves, resign.  If there is a legal move that results
in checkmate, do that move.  Otherwise, for each legal move, consider
each of the possible moves by the opponent.  Each one of those results
in a new board, which can then be examined by our chess-playing
function!  It may seem surprising that a function can be defined this
way, but such ``recursive'' functions are common in mathematics, the
most common being the factorial function, defined by the equation 
\begin{displaymath}
n! = 
\left\{
        \begin{array}{ll}
                1              & \mbox{if } n = 0 \\
                n \cdot (n-1)! & \mbox{otherwise}
        \end{array}     
\right.
\end{displaymath}       
We will use recursive functions extensively in the rest of this book.

The problem with this naive chess-playing function is that there are
too many moves to consider.  While the function can, \emph{in principle},
select the best move to play next, \emph{in practice}, the computation
would take too much time---more than the expected lifetime of the
universe with current technology.  But in fact, \textit{Deep Blue}
did play like this, only it used a massively parallel computer so that
it could consider many moves at the same time, and in most cases it
considered only six to eight moves in the future.  In practice,
\textit{Deep Blue} is a complicated function, with a large
definition.  But in principle, it is just a mathematical function,
just like the programs we will study in this book. 

\begin{ExerciseList}
\Exercise This is the first exercise
        \Question what do you think?
        \Question are you sure?

\Exercise Another exercise
        \Question really?
\end{ExerciseList}      
        
%%% Local Variables: 
%%% mode: latex
%%% TeX-master: "book"
%%% End: 
