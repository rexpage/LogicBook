\chapter{Binary Notation}

%\textit{binary notation, basic sequence operations (so we can talk about lists of bits), \dots}

Numbers are mathematical objects with certain properties,
and they come with a set of operators, such as addition
and multiplication, that produce new numbers from numeric
operands.
As mathematical objects, numbers are ephemeral.
You can't really get your hands on them.
They are figments of the imagination.

However, numbers are useful and to deal with them,
we need to represent them in some way, usually with decimal numerals.
For example, the symbol 144 stands for the number
of eggs in a dozen cartons of eggs.
The symbol 1215 stands for the number of
years between the twenty-seventh year of the reign
of Caesar Augustus and the signing of the Magna Carta.

However, those are not the only symbols we use for the numbers.
The symbols CXLIV and MCCXV stand for the same numbers.
So do the symbols $90_{16}$ and $4BF_{16}$.
The other symbols are Roman numerals and hexadecimal numerals.
The symbols 144 and 1215 are decimal numerals,
which is the representation
people usually turn to when they do arithmetic.

The decimal representation is, in fact, so embedded in
our experience and practice that we often confuse
the symbol with the number.
In fact, the dictionary lists ``number'' and ``numeral'' as synonyms.
Usually, there is no harm in considering them to be the same thing,
but we are going to use numerals
to do arithmetic in a mechanized way, and we will
be careful to separate numbers as mathematical
objects from the symbols we use to represent them.
We will refer to the mathematical object as a ``number''
and to the symbol representing it as a ``numeral''.

Let's think about how we interpret a decimal numeral as a number.
Take the numeral 1215, for example.
Each digit in the numeral has a different interpretation.
The first digit is the number of thousands in the number
that 1215 stands for. The second tells us the number
of hundreds, then the tens, and finally the units.
The following formula is a way to express this interpretation.

\begin{center}
$1 \times 10^3 + 2 \times 10^2 + 1 \times 10^1 + 5 \times 10^0$
\end{center}

This formula shows what the individual digits in the numeral stand for,
and gives us a leg up up on figuring out other kinds of numerals.
The digits in the hexadecimal numeral have a similar meaning,
but with a different basis. Decimal numerals are based on
powers of ten, but hexadecimal numerals are based on powers of sixteen.

The system of decimal numerals calls of ten different symbols for digits,
and we use the symbols 0, 1, 2, \dots 9 for this purpose.
The hexadecimal system calls for sixteen different symbols,
and we use 0, 1, 2, \dots 10, A, B, C, D, E, F to represent them.
The digits stand for the customary numbers (0 for zero, 2 for two,
and so on).

The letters stand for the numbers beyond nine.
There are no conventional squiggles for digits beyond 9.
The letters A, B, \dots F were chosen for this purpose more-or-less arbitrarily.
So, in hexadecimal notation, ``A'' stands for ten, ``B'' for eleven,
and so on up to ``F'' for fifteen. That leads to the following
formula to express the meaning of the hexadecimal numeral $4BF_{16}$.
\begin{center}
$4 \times 16^2 + 11 \times 16^1 + 15 \times 16^0$
\end{center}

\begin{aside}
Perhaps you noticed a subtle confusion in the formulas we use
to explain the meaning of numerals. At first, we claim that
1215 is merely a symbol standing for a mathematical object.
And, we claim that the digit 2 is merely a symbol standing
for the number of items in a pair, along with similar
claims for the digits 1 and 5. Then, we use those symbols
in the formula $1 \times 10^3 + 2 \times 10^2 + 1 \times 10^1 + 5 \times 10^0$
as if they were numbers.

There is some slight of hand going on here.
We are trapped by our terminology.
Numbers as mathematical objects are figments of our imagination,
but when we write formulas, we have to choose some symbols to
represent them.
So, in the formula $1 \times 10^3 + 2 \times 10^2 + 1 \times 10^1 + 5 \times 10^0$,
we use the symbols 1, 10, 3, 2, 5, and 0 as if they were numbers.
But, in the numeral 1215, the symbols 1, 2, and 5 are not numbers.
They are symbols standing for numbers.

It's even worse with the hexadecimal numeral $4BF_{16}$
and the formula $4 \times 16^2 + 11 \times 16^1 + 15 \times 16^0$.
In the formula we have rewritten the symbol ``B'' as the decimal numeral 11
and the symbol ``F'' as the decimal numeral 15.
And, we've had the temerity to pretend that symbols
in the formula are numbers when they are really decimal numerals,
just as we did in the formula that was supposed to explain
the meaning of the decimal numeral 1215.

Furthermore, we've really mixed things up in the numeral
$4BF_{16}$ because the ``4BF'' part is in hexadecimal notation
and the ``16'' part is a decimal numeral indicating that we are
to interpret the digits in base sixteen rather than the conventional base ten.

Try to get your head around this slight of hand.
We're more-or-less stuck with it. Figments of our imagination have
to be materialized, somehow, if we are going to talk about them.
\caption{Digits as Numbers}
\label{aside-digits-as-numbers}
\end{aside}

Formulas like $1 \times 10^3 + 2 \times 10^2 + 1 \times 10^1 + 5 \times 10^0$
convert numerals to numbers.
No doubt you could use this example to construct the appropriate formula
for any given numeral. Base 10 numerals for sure, and hex numerals, too,
and probably any other given base.

We'll say more about to converting numerals to numbers later,
but what about going the other direction, converting
numbers to numerals.
Suppose someone gives you an operator that is supposed to
mechanize the conversion of a decimal numeral to a number.
If ``dgts'' were the name of the operator, the formula
``(dgts 1215)'' would deliver the sequence ``(5 1 2 1)''.

\begin{aside}
\emph{What! The function delivers the digits backwards! Why is that?}

Of course, the function could have delivered the digits in the
customary order, but we have a reason for encoding the numeral
backwards in the sequence that we are using to represent it.
The reason will become apparent soon, but for now, just get used
to it. We write numerals like ``1215'' in the usual way, but
the function we've been given delivers them in the form of
a sequence ``(5 1 2 1)'' in which the digits appear in the
reverse order.

Besides being backwards, the elements in the sequence are numbers,
not symbolic digits.
That is another part of our slight of hand.
We are using numbers to represent the digit symbols.
We could convert them to pure symbols,
but at this point, we are better off leaving them in the form of numbers
because we will want to use them as numbers later.

By the way, the sequence notation ``(5 1 2 1)'' is the symbol we use
to describe the sequence, but the sequence itself is another kind of mathematical object.
That is, it's another figment of our imagination.
The ``real'' object is ephemeral in the same sense as numbers.
\caption{Numerals as Sequences \dots Backwards}
\label{numeral-as-sequence}
\end{aside}

Before you make heavy use of the operator, you will want to check it out.
You might try it on a few numbers, for a start.
\begin{lstlisting}
(check-expect (dgts 1215) '(5 1 2 1))
(check-expect (dgts 1964) '(4 6 9 1))
(check-expect (dgts 12345) '(5 4 3 2 1))
(check-expect (dgts 0) '())
\end{lstlisting}

Wait a minute!
Why does (dgts 0) deliver the empty sequence instead of ``(0)''?
That's another little trick.
Besides delivering the digits in reverse order, leading zeros are always omitted.
We could write the numeral ``1964'' with as many leading zeros as we like.
The numerals ``01964'' and ``000001964'' also stand for the number 1964.
Those numerals would be ``(4 6 9 1 0)'' and ``(4 6 9 1 0 0 0 0 0)'' in
the sequence form of the dgts operator.

However, dgts never includes any leading zeros in the numerals it delivers.
It leaves them all off, even for the number zero.
That's why (dgts 0) is nil.
Of course (dgts 012345) would be the same as (dgts 12345), too,
because ``012345'' and ``12345'' stand for the same number.
Both of those formulas would deliver the sequence ``(5 4 3 2 1)''.
Remember: dgts operates on numbers, not numerals.

The computer interprets the decimal numeral in the formula (dgts 012345)
as a number and converts it to a mathematical object. That is, a number.
What does that number look like? None of your business.
That's the computer's business.
It has it's own way of dealing with numbers.
Later, we'll study the way most computers do this,
but we will assume that the computer has some way of turning
numerals into whatever form it uses to deal with numbers.

Now that we've run a few sanity checks, we want to get down to some serious testing.
That means big batches of automated tests on random data.
Coming up with automated tests calls for a little more thought.
Let's start small. How about the last digit in a decimal numeral?
What mathematical formula would deliver the last digit in a
decimal numeral, given an arbitrary, positive integer $n$?

The last digit in a decimal numeral is the remainder when you divide
the number by ten. The formula that converts a numeral to a number
makes that clear.
\begin{center}
$1 \times 10^3 + 2 \times 10^2 + 1 \times 10^1 + 5 \times 10^0$
\end{center}

Each of the terms in the formula is a product of a power
of ten with another number. A power of ten is, of course,
a multiple of ten, so none of the terms contribute to the remainder
when dividing by ten. None of them, that is, except the last one.
It does not have a factor of ten in it because the ten is raised to the
power zero, and anything to the power zero is one, which is not a multiple of ten.

So, to get the last digit in the numeral, all we need to do is to
compute the remainder in the division of the number by ten.
Computing the remainder in a division operation
is what the ``mod'' function does (see page \pageref{mod-function}).
The following formula uses mod to check the last digit in the numeral
that the dgts operator delivers for the number $n$.
\begin{lstlisting}
(= (first (dgts n)) (mod n 10))
\end{lstlisting}

Since dgts delivers the digits backwards,
``(car (dgts $n$))'', the first digit in the sequence,
is the last digit in the numeral.
The formulas checks to make sure that digit is ``(mod $n$ 10)'', the remainder when
dividing $n$ by 10.

We can use the DoubleCheck facility of Dracula to run this test on a batch of random numbers.
We need to be careful not to allow zero to pop up in the testing
because (dgts 0) is nil, so there is no first digit to check.
Besides, we've already completed the testing of (dgts 0) in our sanity checks.
We can avoid retesting zero by adding one to a random natural number.
That produces a random, non-zero, positive integer.
\begin{lstlisting}
(defproperty dgts-last-digit-tst
  (n  :value (+ 1 (random-natural))) ; avoid zero
  (= (first (dgts n))
     (mod n 10)))
\end{lstlisting}

That takes care of testing the last digit, but what about the others?
We can do something about those by observing that if we discard the
remainder when we divide by ten, we get a number with the same digits,
except that the last one is missing.
Since we've already tested the last one, we don't need to worry about that,
but we need to find a way to convert our observation about the other digits into a test.

One way to do that is to make sure that dgts delivers the right digits when
we apply it to $n$ divided by ten when we discard the remainder.
The intrinsic function ``floor'' does this kind of division.
It delivers the quotient (without the remainder) when dividing its first argument (the dividend)
by it second argument (the divisor). The following formula implements our test.
It checks to make sure the digits after the first one in the sequence dgts delivers for $n$
are the same as the digits in the sequence dgts delivers for the quotient in $(n \div 10)$,
that is (floor $n$ $10$).
\begin{lstlisting}
(equal (rest (dgts n))
       (dgts (floor n 10))))
\end{lstlisting}

As with the test of the last digit, we can run a batch of tests based on our rest-of-the-digits
observation by defining a DoubleCheck property.

\begin{lstlisting}
(defproperty dgts-other-digits-tst
  (n  :value (+ 1 (random-natural))) ; avoid zero
  (equal (rest (dgts n))
         (dgts (floor n 10))))
\end{lstlisting}

It would be nice to run these tests, but ``dgts'' is not an intrinsic operator.
We have to provide a definition for it.
To do that we use the ``defun'' command, which is similar to defproperty,
but without any value specifications. The definition will be inductive,
of course, using the same ideas as the tests,
and will follow the three C's of Figure~\ref{fig:inductive-def-keys}
(page \pageref{fig:inductive-def-keys}).
\begin{center}
\begin{tabular}{lp{3.5in}}
\emph{Complete} & Either the number is zero or it isn't. Those are the only two cases.
Our definition will have separate formulas for these two cases. \\
\emph{Consistent} &  The cases do not overlap, so there is no chance for inconsistency.\\
\emph{Computational} & In the non-inductive case (when $n$ is not zero), the argument of
``dgts'' is divided by ten which makes it closer to zero (the non-inductive case).
\end{tabular}
\end{center}

The definition takes the following form.
\begin{lstlisting}
(defun dgts (n)
  (if (zp n)
      nil                          ; n is zero
      (cons (mod n 10)             ; n non-zero
            (dgts (floor n 10)))))
\end{lstlisting}

This definition uses an operator, ``zp'', that we have not discussed before.
It delivers true if its argument is the natural number zero
(or, if it isn't a natural number at all),
and false otherwise.

You might think a comparison with zero using the ``='' operator
would work equally well, but it doesn't because we need to select
the non-inductive formula if the argument is not the kind of
number we expect. For example, the formula (= $n$ 0) would have the
value false if $n$ were 3/2, but 3/2 his a fraction.

We have not accounted for fractions in our analysis.
Our formula for the non-zero case assumes that $n$ is a natural number.
So, if $n$ is not a natural number,
we bail out and deliver the empty list.

If you put the definition of dgts at the beginning of a program
and import the ``testing'' and ``doublecheck" facilities, you
can enter the tests and run them using Dracula.
You can also enter formulas in the command pine to compute
decimal numerals for any natural numbers you choose.

The ``dgts'' function provides a way to produce a decimal
numeral, given a number. How about going in the other direction?
Given a decimal numeral, produce the corresponding number.
You already know that the formulas look like these:
\begin{center}
$1 \times 10^3 + 2 \times 10^2 + 1 \times 10^1 + 5 \times 10^0$ --- for 1215
$1 \times 10^2 + 4 \times 10^1 + 4 \times 10^0$ --- for 144
\end{center}

But, instead of thinking of formulas like that, let's think of properties
that a function converting decimal numerals to numbers would have.
Since the dgts function produces numerals represented as sequences
of decimal digits (numbers in the range 0 to 9), and in the reverse
of the usual order (that is, with the units digit first, then the tens
digit, then hundreds, and so on), let's assume that numerals will
take that form.

Let's call the decimal-numeral-to-number converter ``num10''.
We know that (num10 nil) must be zero because (dgts 0) delivers
zero, and we are trying to convert numerals produced by dgts
back the the numbers they came from.

How about a one-digit numeral $(x_0)$.
The equation in that case would be (num10 ($x_0$)) = $x_0$.

If there are two or more digits, the numeral would take the form
($x_0$ $x_1$ \dots $x_{n+1}$).
Then, the equation would be
\begin{center}
(num10 ($x_0$ $x_1$ $x_2$ \dots $x_{n+1}$)) = $(x_0 + x_1 \times 10^1 + x_2 \times 10^2 + \dots x_{n+1} \times 10^{n+1})$
\end{center}

Since all of the terms in the sum include a factor of ten, except the first term,
we can factor ten out of all of the terms beyond the first.
Factoring the formula in this way produces a new equation.
\begin{center}
(num10 ($x_0$ $x_1$ $x_2$ \dots $x_{n+1}$)) = $(x_0 + 10 \times (x_1 \times 10^0 + x_2 \times 10^1 + \dots x_n \times 10^n))$
\end{center}

But, the sequence ($x_1$ $x_2$ \dots $x_{n+1}$) is also a decimal numeral.
And, the value is stands for it $(x_1 \times 10^0 + x_2 \times 10^1 + \dots x_n \times 10^n)$,
which is the value num10 should deliver, given the numeral ($x_1$ $x_2$ \dots $x_{n+1}$).
That is, the following equation holds.
\begin{center}
(num10 ($x_1$ $x_2$ \dots $x_{n+1}$)) = $(x_1 + x_2 \times 10^1 + \dots x_{n+1} \times 10^n))$
\end{center}

Observe that this formula for (num10 $(x_1 x_2 \dots x_{n+1})$) is identical
to the factor multiplied by ten in the previous equation.
Therefore, we can rewrite that equation as follows.
\begin{center}
(num10 ($x_0$ $x_1$ $x_2$ \dots $x_{n+1}$)) = ($x_0 + 10 \times$ (num10 ($x_1$ $x_2$ \dots $x_{n+1}$)))
\end{center}

Now, we have an inductive equation that delivers the right value for
numerals with two or more digits. Furthermore, the same formula works
for one-digit numerals because (num10 nil) is zero.
\begin{center}
(num10 ($x_0$)) = ($x_0 + 10 \times$ (num10 nil))
\end{center}

We now have one equation that covers all numerals with one or more digits.
Furthermore, we by the \{cons\} axiom (see page \pageref{cons-axiom-informal}):
\begin{center}
($x_0$ $x_1$ \dots $x_{n}$) = (cons $x_0$ $(x_1 \dots x_n)$)
\end{center}

Therefore, by the axioms relating the functions first, rest,and cons (see page \pageref{first-rest-cons}),
we arrive at the following equations.
\begin{center}
$x_0$ =  (first ($x_0$ $x_1$ \dots $x_{n}$))
$x_1$ \dots $x_{n}$) = (rest ($x_0$ $x_1$ \dots $x_{n}$))
\end{center}

This, along with using $xs$ as a shorthand for ($x_0$ $x_1$ $x_2$ \dots $x_{n+1}$)
allows us to rewrite the equation for non-empty numerals in prefix notation as follows.
\begin{center}
(num10 $xs$) = (+ (first $xs$) (* 10 (num10 (rest $xs$))))
\end{center}

Now we have the basis to apply the rule of the three C's 
(page \pageref{fig:inductive-def-keys}) to define num10.

\begin{lstlisting}
(defun num10 (xs)
  (if (consp xs)
      (+ (first xs)               ; non-empty numeral
         (* 10 (num10 (rest xs))))
      0))                         ; empty numeral
\end{lstlisting}

\todo{do inductive proof that num10 delivers the right number in the usual equation-based format 
... basically it repeats the above, sort-of scattered out, derivation of the def'n}

\todo{inductive proof that (num10 (dgts n) = n ... discuss why (dgts (num10 xs)) is not necessarily xs
because of the leading zeros problem?? ... prove (dgts (num10 (drop-leading-zeros xs))) = (drop-leading-zeros xs)?}

\todo{do it all over in binary, maybe leave the binary-to-num direction as an exercise}

%%% Local Variables:
%%% mode: latex
%%% TeX-master: "book"
%%% End:
