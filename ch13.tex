\chapter{Sharding with Facebook}

Facebook is a website that redefined social networking.  As of this writing in 2011, just seven
years after its launch in 2004, it has grown to connect more than 600 million users---about twice
the total population of the United States or nearly one out of every 11 people in the entire world.

Having so many users poses tremendous technical challenges, and Facebook's success is partly due to
their ability to handle these problems.  Facebook may look like an ordinary website, but behind it
lies incredible technical innovation.  In this chapter, we'll discuss some of these innovations.

\section{The Technical Challenge}

To understand the challenges facing Facebook, let's consider just one of their main features.  Facebook
users can post short status updates, and they can view the status updates posted by their friends.
Implementing this feature requires two pieces of information for each user:
\begin{enumerate}
	\item a list of each user's friends, and
	\item a list of each user's status updates.
\end{enumerate}
That amounts to 1.2 \emph{billion} lists, and it's not unusual for a list to have hundreds of items.

Traditionally, this data would be stored in a database using two ``tables,'' one for the status updates,
and one for the friends.  Figure~\ref{facebook-tables} shows what these tables could look like.  Databases
can manage tables, taking care of the details of inserting, deleting, and modifying rows.  They also
support sophisticated queries that can retrieve information from the tables.  For example, Facebook could
use the following query to determine what status updates to display when a specific user, say John, logs in:
\begin{verbatim}
	SELECT   s.User, s.Time, s.Status
	FROM     Friends f, Statuses s
	WHERE    f.User='John'
	  AND    f.Friend = s.User
	ORDER BY s.Time DESC
\end{verbatim}
This approach would work very well if Facebook had a small number of users, but it simply does not scale to
600 million users.  The problem is that the database cannot process this query quickly enough to keep up with
the demand from Facebook's users.

Facebook is not the only company that has this problem.  For example, Amazon offers over ten
million items for sale.  It encourages customers to write reviews for each item, and it keeps a history
of purchases made by its customers so it can make suggestions for products to buy.  This information
could easily be stored in tables in a traditional database, but retrieving it would take too long.

In fact, this is a problem faced by many Web 2.0 companies.  By ``Web 2.0'' we mean a website that it allows users, as well as website owners, to produce content for the website.
This can be directly in the form of passages, as in
Facebook updates or Amazon reviews, or indirectly, as in Amazon recommendations.  
When a Web 2.0 company is successful,
its users produce so more content than traditional databases can handle.

\begin{figure}
	\begin{center}
		\begin{tabular}[t]{ll}
			\hline
			\multicolumn{2}{c}{\textsc{Friends}} \\
			\hline
			\multicolumn{1}{c}{\textbf{User}} & \multicolumn{1}{c}{\textbf{Friend}} \\
			\hline
			John  & Sally \\
			John  & Mary \\
			Mary  & Sally \\
			Sally & David \\
			\hline
		\end{tabular}
		\hspace{.5in}
		\begin{tabular}[t]{ll>{\raggedright}p{1in}}
			\hline
			\multicolumn{3}{c}{\textsc{Statuses}} \\
			\hline
			\multicolumn{1}{c}{\textbf{User}} & \multicolumn{1}{c}{\textbf{Time}} & \multicolumn{1}{c}{\textbf{Status}} \\
			\hline
			John  & Apr 21, 2011, 10:27 am & Checked into Starbucks in Norman. \tabularnewline
			Sally & Apr 21, 2011, 10:29 am & Saw \emph{Battle: Los Angeles} last night.  What a waste! \tabularnewline
			Mary  & Apr 21, 2011, 10:32 am & Is anybody going to the carnival this weekend? \tabularnewline
			Sally & Apr 21, 2011, 10:33 am & Looks like the fires are getting closer to our house.  Thinking about evacuating. \tabularnewline
			\hline
		\end{tabular}
	\end{center}
	\caption{Facebook tables for storing status updates}
	\label{facebook-tables}
\end{figure}

\section{Stopgap Remedies}

\subsection{Caching}

The basic problem is that traditional databases take too long to retrieve the information needed for
a user's welcome screen.  One way to address this is to limit the number of queries that the
database needs to carry out.  
For example, if a user checks her home screen several times in a short time span,
the computer can simply remember the previous results, instead of asking the database to
retrieve them again and again, each time the user checks her screen.

That's called \emph{caching} the data.  A cache is a key/value store, like the search trees of chapter \ref{search-trees} or the hashtables of chapter \ref{hash-tables}.
When a query arrives, the database first checks the cache to see of the information is already there.
If it is, it is simply reused.  Otherwise, the database executes the query,
and puts the results in the cache, so they can be reused later.

Caching is useful when three conditions are met.  
First, putting information in the cache and retrieving must be faster than ordinary database operations. 
Much faster, so there is a significant gain when the results are in the cache and only a tiny delay
when they're not.  Second, interactions with the database must frequently repeat the same transactions,
so that results stored in the cache are often reused.  
Third, retrieval, rather than update, must be the dominant type of database transaction.
Updates can make the data in the cache inconsistent with information in the database,
forcing it to be retrieved anew, just as it would be if there were no cache at all.
So, if updates are the dominant transaction, caching is a waste of time and effort.

Caching is used throughout the web.  It is especially successful in storefront applications, where
database queries are returning details about a particular product.  
Storefronts often hundreds of thousands of products,
but only a handful of them are really popular.
So, there are frequent requests for the same information, which makes the cache effective.

Caching worked well for Amazon, at least before product reviews and recommendations became prevalent
in their product pages.  
But caching cannot work well for Facebook.  The problem users look at their
welcome pages on an individual basis, and they make frequent postings, so retrieving information
does not dominate the pattern of transactions the way it does with Amazon product pages.
Besides that, one update on Facebook can trigger changes in many pages of the website.
This is typical in Web 2.0 applications, and this need for frequent updates of customized information for every user eliminates the advantages of caching. Another solution is needed.

\subsection{Sharding}

\emph{Sharding} splits the database into many different databases.
For example, John's Facebook friends and status updates may be stored in machine J, while
Mary's friends and status updates are stored in machine M.  
Machines, like J and M, which store just a portion of the data are called \emph{shards}. 
Because the database is stored on many different computers, 
no one computer has to shoulder the entire load. 
That's the upside. 
The downside is that it makes it harder to automate transactions with the database.
Programmers have to specify how to distribute individual queries across all of the many computers
that may be involved in resolving the query.

To see how this might work, 
suppose that the computer needs to generate John's welcome page.  
The first step is to find John's users, which can be done by executing a query on machine J:
\begin{verbatim}
	SELECT   f.Friend
	FROM     Friends f
	WHERE    f.user='John'
\end{verbatim}
The query returns John's friends, Sally and Mary.  
The next step is to find Sally's and Mary's status updates.
That leads to the the following queries:
\begin{verbatim}
	SELECT   s.Time, s.Status
	FROM     Statuses s
	WHERE    s.User='Sally'
	ORDER BY s.Time DESC

	SELECT   s.Time, s.Status
	FROM     Statuses s
	WHERE    s.User='Mary'
	ORDER BY s.Time DESC
\end{verbatim}
Of course, the first query should run on machine S, 
while the second query should on machine M.  
The final step is to combine the results from the two queries,
and then the results need to be merged in such a way
that the combined list stays in reverse chronological order.

Queries like this show one of the shortcomings of sharding.  
Notice that each query retrieves results from only one table.  
The reason is that the related records are not necessarily stored in the same shard.  
In this particular example, the information needed to answer the query was 
distributed across shards J, S, and M.  
So the program had to collect the information from all the different sources and then combine it.  
Clearly, using sharding is much more difficult than keeping all the information in one place,
as in a traditional database.

It also suffers from uneven distribution.  
What if, for example, one of the shards ends up with too many records?
That shard could be split into pieces.
For example, the system could
split the Ms shard into to shards, one for Ma-Mp, and another for Mq-Mz.  
That seems like a good idea, but in practice splitting shards
is complicated because the software on all the computers that access the data
need to be modified, so that they can find the shard that contains the information they're looking for.

\section{The Cassandra Solution}

Faced with these difficulties, Facebook engineers developed a solution that  
retained the benefits of sharding, but avoided some of the difficulties.  
The goal was to make easy to split a shard into multiple pieces, 
and to hide from the software the complexity of sharding.

Cassandra, the solution they devised, combines features from Amazon's Dynamo project and Google's BigTable.  From Dynamo, Cassandra borrows the idea of a replication ring, 
and from BigTable a data model.

Cassandra's data model groups records into into different tables.  
Each record in a table is identified with a key.  
The key of key must be unique in a given table, but the same key may be used in different tables.
Each record consists of one or more key/value pairs.

For example, John's friends may be stored in a record like the one shown in Figure~\ref{users-table}.
The important thing is that a program can retrieve 
all of John's friends by requesting the single record with ID John.  
(Note: Different records in a Cassandra table may have different keys.)

Once John's friends are known, it is necessary to retrieve their status updates.  
This can be done by looking for the records in the status table that have the appropriate record IDs.  Figure~\ref{status-table} shows what the status table could look like.

The table structures we have presented assume that all fields will fit in a single record.
That is, we assume that a single record can hold all of a user's friends, or all of a user's status updates.
Cassandra tables are designed to support thousands of fields.  
This will be enough for most users, but not for the heaviest users of Facebook.  
To deal with the heaviest users, Facebook can reuse the same idea with column names.  
To support an arbitrary number of friends or status updates, 
the values can be spread across multiple records, with IDs such as John1, John2, etc.

\begin{figure}
	\begin{center}
		\begin{tabular}[t]{lll}
			\hline
			\multicolumn{3}{c}{\textsc{Friends}} \\
			\hline
			\multicolumn{1}{c}{\textbf{Record ID}} & \multicolumn{1}{c}{\textbf{Friend1}} & \multicolumn{1}{c}{\textbf{Friend2}} \\
			\hline
			John  & Sally & Mary \\
			Mary  & Sally & \\
			Sally & David & \\
			\hline
		\end{tabular}
	\end{center}
	\caption{Storing friends lists in Cassandra}
	\label{users-table}
\end{figure}

\begin{figure}
	\begin{center}
		\begin{tabular}[t]{lp{.8in}p{.8in}>{\raggedright}p{1in}>{\raggedright}p{1in}}
			\hline
			\multicolumn{5}{c}{\textsc{Statuses}} \\
			\hline
			\multicolumn{1}{c}{\textbf{Record ID}} & \multicolumn{1}{c}{\textbf{Time1}} & \multicolumn{1}{c}{\textbf{Time2}}
				& \multicolumn{1}{c}{\textbf{Status1}} & \multicolumn{1}{c}{\textbf{Status2}} \\
			\hline
			Sally & Apr 21, 2011,\hfill\break10:29 am & Apr 21, 2011,\hfill\break10:33 am &
				Saw \emph{Battle: Los Angeles} last night.  What a waste! &
				Looks like the fires are getting closer to our house.  Thinking about evacuating. \tabularnewline
			\hline
		\end{tabular}
	\end{center}
	\caption{Storing status updates in Cassandra}
	\label{status-table}
\end{figure}

The upshot is that the workflow for retrieving information from Cassandra 
is very similar to the workflow for sharding, but with a major difference.  
The queries that are generated do not need to be aware of which shard contains the information they need.

In fact, Cassandra relies on sharding both for performance and for replication.  
The key innovation is that the shards are arranged in a ring.  
For simplicity, assume that the shards are labeled A, B, C, \dots, Z.  
The ring arrangement means that each shard is connected to the next, 
and eventually the last shard is connected to the first.  
For example, A could be connected to B, B to C, and so on, until Z is connected back to A.

The records are arranged in an order determined by a hash function.  
The record's key is hashed, and the resulting hash value is mapped to the shard labels (A-Z in the example).
That is not to say that the hash function can only take on the values A, B, \dots, Z.
Rather, it is that all hash values up to A are mapped to shard B, 
those between A and B to shard C, and so on.

The hash function and mapping of hash values to shards is known to every shard.  
So a program can ask any of the machines to retrieve a given value.  
If the shard does not have the value, it can easily determine which is the correct shard, 
and it forwards the request to that shard.

The ring makes it easier to rebalance the shards, in case one of them becomes too large.  
Suppose, for example, shard B gets too large.  
To balance it, a new shard, say BM, is created and inserted between B and C.  
During the insertion process, B sends all of its records between BM and C to shard BM.  
When this process completes, all shards are notified of the new
shard's existence, and shard BM joins the ring.  
This can happen without the knowledge of any programs that are retrieving data from the system.

Cassandra also uses the ring for replication.  
What this means is that records that should be stored in A are \emph{also} stored in B and C.  
This is important, because computers and disks can fail, 
but if shard A should fail, there are two more copies of its data.  
It can also serve to improve performance during spikes.  
If shard A becomes busy, shards B and C can take over some of the load.

Replication complicates the splitting of shards somewhat.  
When shard B is split into B and BM, for example, this also affects shards C and D.  
Shards C and D replicated all the records for shard B.  
Now this needs to be restructured, so that B's records are replicated in shards BM and C.  
Moreover, C and D should replicate the records for shard BM.  
Effectively, this means that shards C and D need to participate in the insertion of BM into the ring.  
In the case of C, it simply needs to know that some of the records 
it replicated for B are now associated with BM instead.  
For D, it means that some of the records it was replicating
on behalf of B can now be forgotten, and the rest need to be associated with BM.  
Finally, BM needs to receive not just its share
of B's records, but all of B's records, so that it can replicate B's remaining data.

\section{Summary}

Web 2.0 applications bring up many challenges due to scaling.  
These challenges go beyond what traditional database solutions can offer.  
So, many of the leading Web 2.0 companies have developed custom solutions based on the idea of sharding.  Cassandra, Facebook's solution, represents the current state of the art.  
Fortunately, Facebook decided to make Cassandra available to the
programming community via an open source process.  
Programmers can download Cassandra from \url{http://cassandra.apache.org}
and use it to develop new software.

%%% Local Variables:
%%% mode: latex
%%% TeX-master: "book"
%%% End:
