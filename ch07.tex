\chapter{Multipliers and Arbitrary-Precision Arithmetic}
\label{ch:multipliers}

%%\textit{shift and add multipliers, \dots}

Chapter \ref{ch:adders} discussed circuits for adding binary numerals of a
fixed word length.
The ACL2 model (Figure \ref{fig:adder}, page \pageref{fig:adder})
assumed that both numerals had exactly $w$ bits, $w$ being the word size.
The circuit diagram (same figure) reflected this assumption by
showing $2w$ input wires ($w$ lines for each numeral) and
$w$ output wires for the bits of the numeral representing the sum.

The diagram has an additional input wire for the carry bit
coming into the adder (normally a zero bit, unless the circuit is being
used for some sort of multi-word arithmetic) and an additional output wire
for the output carry bit. The output carry bit is normally ignored in single-word arithmetic when one summand is positive and the other is negative,
but used to detect overflow\footnote{Overflow occurs
when the sum of the two input numerals fails
to fall into the range of integers representable in $w$-bit, twos-complement
arithmetic.}
when they have the same sign.

The ACL2 model received the input carry as its first argument
and the two input numerals as lists of length $w$.
It delivered the output as a list of two elements,
the first element being a list of $w$ sum bits,
and the second element being the carry out.
The model, like the circuit diagram, did not allow for input numerals
of differing lengths.

That is the nature of physical circuits.
They have a fixed number of wires and gates.
Software has no such constraint.
A software model for adding binary numerals can accept
numerals of any length, and the two numerals need not have the same length.
The numeral representing the sum would then have as many bits as
might be required to represent the sum of the numbers
represented by the two input numerals.

An adder expressed in software that is able to deal with numerals
of any length is sometimes called a ``bignum'' adder.
It performs arbitrary-precision arithmetic
rather than arithmetic on fixed word sizes.
The twos-complement scheme used with arithmetic on fixed-size words
does not work with arbitrary-precision arithmetic, so negative numbers
have to be accounted for in some other way.

In this chapter, we will discuss arbitrary-precision arithmetic
for non-negative integers,
but not for negative ones.
We think adding the complexity of negative numbers detracts from the clarity
of the presentation without adding much in the way of intellectual
depth. It also provides a fertile ground for small projects that
can be used for practice with the concepts, so we leave it to that realm.

\section{Bignum Adder}
\label{sec:bignum-adder}

To begin, let's see what it takes to convert our ACL2 model for
the ripple-carry adder to software that performs addition on
binary numerals with an arbitrary number of bits,
representing, of course, numbers of unlimited size.

One way to start the discussion is to work out a way to increment a
binary numeral by one. That is, given a binary numeral $x$ representing
the number $n$, we seek a way to compute the numeral for the number $n+1$.

\begin{tabular}{ll}
(add-1 $x$) = numeral for (+ (nmb $x$) 1)           & \\
\end{tabular}

Following our usual practice when we are trying to define an operator,
we assume that someone has already defined it,
and all we have to do is to write some equations that it
would have to satisfy if it worked. If we are able
to come up with equations that are consistent, comprehensive, and computational
(Figure~\ref{fig:inductive-def-keys}, page \pageref{fig:inductive-def-keys})
we will have at least defined some operator. But not just any opertor.
We will have defined the only operator
that satisfies those equations).

A particularly simple situation occurs when the numeral to be incremented
has no bits in it. The interpretation we settled on
in Chapter \ref{ch:binary-numerals} is that
the empty numeral stands for the number zero
(definition of nmb, page \pageref{nmb-defun}).
So, incrementing the empty numeral should produce a numeral for the number 1.

\hspace{1cm}(add-1 nil) = (list 1)  \hfill \{add1nil\}

Another simple situation occurs when the low-order bit in the
numeral to be incremented is a zero.
In that case, the numeral for the incremented number is
just like the numeral to be incremented, except that its
low-order bit is a one rather than a zero.

\hspace{1cm}  (add-1 (cons 0 x)) = (cons 1 x)    \hfill \{add10\}

At this point, we have equations to cover all numerals
that have either no bits at all or whose low-order bit
is zero. If we can work out an equation for numerals
with a low-order bit of one, our equations will be comprehensive.

To do this, let's think about the
low-order bit of the incremented numeral.
Since adding a one-bit to a one-bit produces a sum-bit
of zero and a carry-bit of one
(Figure~\ref{fig:half-adder}, page \pageref{fig:half-adder}),
we conclude that the low-order bit of the incremented numeral
is zero.

But, what about the carry-bit? What do we do with that?
It will need to be added to the higher-order bits of
the input numeral. But, that is just a matter of incrementing
the higher-order bits by one.
The higher-order bits are, themselves, a numeral,
and because of our standard assumption that
someone has already defined the function we need,
we can just use it to increment that numeral.
That is, we can write an inductive equation that the add-1 operator
must satisfy.

\hspace{1cm}  (add-1 (cons 1 x)) = (cons 0 (add-1 x))   \hfill \{add10\}

Now we have three equations, and they are
consistent (no overlapping cases) and
comprehensive (all cases covered).
The equations are also computational because the input numeral in
the inductive invocation of add-1 in the only inductive equation
(equation \{add10\}) is a shorter numeral than
the one on the left-hand-side of that equation.
Therefore, the equations define the add-1 operator.
All we need to do now is to translate them to proper ACL2 notation
from the informal sketches we wrote down
while we were thinking about the problem.

\label{add-1-defun}
\begin{Verbatim}
; (add-1 x) = numeral for (+ 1 (numb x)))
; (add-1 nil) = (list 1)                        {add1nil}
; (add-1 (cons 0 x)) = (cons 1 x)                 {add10}
; (add-1 (cons 1 x)) = (cons 0 (add-1 x))         {add11}
(defun add-1 (x)
  (if (and (consp x) (= (first x) 1))
      (cons 0 (add-1 (rest x))) ; add11
      (cons 1 (rest x))))       ; add10
\end{Verbatim}

In doing so, we observe that the \{add1nil\} equation
and the \{add10\} equation can be expressed as one equation because
the ACL2 formula (cons 1 (rest $x$)), which is the proper
translation for the right-hand-side of \{add10\} equation,
also works for the right-hand-side of \{add1nil\} equation
because (cons 1 (rest nil)) has the same value as (list 1).
This observation reduces the definition from three equations to two.
That solves the increment-by-1 problem.

The ripple-carry adder propagated the carry from each bit position
to the next higher-order bit position.
Each bit position involved three input bits
(a carry and one bit from each addend).
Our bignum adder will need to do something similar.
That is, each bit in the sum will depend on the
corresponding bits in the addends and the carry from
the previous, lower-order bit.

We already have the apparatus for this: the full-adder model
(Figure~\ref{fig:full-adder}).
We can use that function, as it was defined
on page \pageref{fig:full-adder},
to add two corresponding bits, $x_n$ and $y_n$ from the addend numerals,
incorporating the carry, $c_n$, from the lower-order bit position.
The full-adder function delivers the sum bit, $s_n$, for the current
bit position and the carry bit, $c_{n+1}$, for the next bit position.
\begin{center}
[$s_n$ $c_{n+1}$] = (full-adder $c_n$ $x_n$ $y_n$)
\end{center}

That analysis provides the basis for
one of the equations for the bignum adder.
That equation covers the situation when there are two,
corresponding bits from the addend numerals to deal with.
In the particular case where the corresponding bits involved
are the low-order bits in the numerals,
that equation would sketch out as follows:

\begin{tabular}{ll}
(add $c_0$ [$x_0$ $x_1$ $x_2$ \dots ] [$y_0$ $y_1$ $y_2$ \dots ]) = [$s_0$ $s_1$ $s_2$ \dots ]   & \{addxy\} \\
where & \\
$[s_0$ $c_1] =$ (full-adder $c_0$ $x_0$ $y_0$) & \\
$[s_1$ $s_2 \dots ] =$ (add $c_1$ $[x_1$ $x_2 \dots ]$ $[y_1$ $y_2 \dots ]$) & \\
\end{tabular}

This equation covers all summands whose numerals have at least one bit.
So, all we need to do to make our collection of equations comprehensive
is to have equations for the cases when one or the other summand is an empty numeral.

If either summand represents zero, which it would if it were an empty numeral,
the sum would be the other summand, with the carry added to it.
We already have a function that produces this result when the carry is one.
When the carry is zero, the result is the same as the summand,
so the add-1 function (page \pageref{add-1-defun}) provides the basis
for adding a carry bit.

\label{add-c-defun}
\begin{Verbatim}
; (add-c c x) = numeral for (+ c (nmb x)))
(defun add-c (c x)
  (if (= c 1)
      (add-1 x)  ; addc1
      x))        ; addc0
\end{Verbatim}

That gives us a way to write the equations when either of the summands is empty.
\begin{tabular}{ll}
(add $c$ $x$ nil) = (add-c $c$ $x$)   & \{add10\} \\
(add $c$ nil $y$) = (add-c $c$ $y$)   & \{add01\} \\
\end{tabular}

Now, with all the cases covered, we can translate our sketched equations
to proper ACL2 notation.

\label{bignum-add-defun}
\begin{Verbatim}
; arbitrary-precision adder (bignum adder)
; (add c0 x y) = numeral for (+ c0 (nmb x) (nmb y))
(defun add (c0 x y)
  (if (not (consp x))
      (add-c c0 y)                                   ; add0y
      (if (not (consp y))
          (add-c c0 x)                               ; addx0
          (let* ((x0 (first x)) ; x is not nil
                 (y0 (first y)) ; y is not nil
                 (a  (full-adder c0 x0 y0))
                 (s0 (first a))
                 (c1 (second a)))
            (cons s0 (add c1 (rest x) (rest y))))))) ; addxy

\end{Verbatim}

Here are a few examples of the bignum add function in operation:

\begin{tabular}{ll}
(add 0 [0 1] [0 1]) = [0 0 1]               &  2 + 2 = 4  \\
(add 0 [0 1 1 1] [1 0 1]) = [1 1 0 0 1]     & 14 + 5 = 19 \\
(add 0 [1 0 0 1 1] [0 1 1]) = [1 1 1 1 1]   & 25 + 6 = 31 \\
\end{tabular}

The property the bignum add function is designed to satisfy
delivers a binary numeral for the sum of
the numbers represented by the two input numerals and the input carry.
The examples show that the property holds for three
particular pairs of addends.
Of course we would like to know that the property
holds for all input numerals.

The following theorem is a formal statement of the property.

\label{bignum-adder-thm}
\begin{Verbatim}
(defthm bignum-add-thm
  (= (nmb(add c x y))
     (+ (nmb(list c)) (nmb x) (nmb y))))
\end{Verbatim}

The ACL2 system succeeds in proving this theorem by induction,
following a similar strategy to one we used in in
Section \ref{sec:adding-negative-numerals}
(page \pageref{sec:adding-negative-numerals}) to prove a theorem
about the ripple-carry adder for fixed-size words.
The upshot is that we now know, to a mathematical certainty,
that the bignum add function delivers the sum of the two input numerals.

\begin{ExerciseList}
\Exercise Do a pencil-and-paper proof by induction of the following
theorem, which says that if the high-order bit in both input numerals is a one,
then the high-order bit of the numeral representing the sum that is delivered
by the bignum add function is also one.
The theorem refers to the fin function (page \pageref{fin-defun}).

\label{bignum-add-hi-order-bit-thm}
\begin{Verbatim}
(defthm bignum-add-hi-order-bit-thm
  (implies (and (= (fin x) 1) (fin y 1))
           (= (fin (add c0 x y)) 1)))
\end{Verbatim}
\end{ExerciseList}

\section{Shift-and-Add Multiplier}
\label{sec:bignum-mult}

The grade-school method of multiplying big numbers proceeds one digit at a time,
from right (low-order digit of the decimal numeral) to left (higher order digits).
The first step is to multiply the entire multiplicand by the low-order digit
of the multiplier.
Then comes next-to-last digit of the multiplier (the tens digit).
In this case the product is written below the one from the first step,
but shifted left one position. After all the products are computed for
all the digits in the multiplier, they are totaled, taking care to keep
the digits lined up according to the shift to the left that occurred at each stage.

Grade-school students learn this procedure without knowing
the algebra behind it. However, we want to use a collection of
equations to specify the product of numerals, so it will be helpful
to know the equations behind the procedure.
Our analysis will use the following nomenclature:

\begin{tabular}{ll}
$x$       & number represented by numeral of multiplier \\
$x_0$     & low-order digit of numeral of multiplier  \\
$y$       & number represented by numeral of multiplicand  \\
\end{tabular}

The procedure for multiplying numerals relies on the following equation,
which grade-school students use to check the correctness of their long-division problems:

\hspace{2mm} $x = (\lfloor x \div d \rfloor \cdot d) + $($x$ mod $d)$ \hfill \{long-division\}

\hspace{2mm} where $\lfloor x \div d \rfloor$ stands for the greatest integer that is $x \div d$ or less

\hspace{2mm} and $(x$ mod $d)$ is the remainder in dividing the integers $x$ and $d$ (modular arithmetic)

We derive the following equation by taking $d = 10$ and
 multiplying both sides of the above equation $y$.

\hspace{2cm} $xy = (\lfloor x \div 10 \rfloor \cdot 10 \cdot y) + ((x$ mod $10) \cdot y)$
\hfill \{long division by 10\} \\

Observe that $(x$ mod $10)$ is the low-order digit
of the decimal numeral for $x$, which is what $x_0$ stands for
in our nomenclature.
So, $(x$ mod $10) \cdot y$ is $x_0 \cdot y$,
(that is, the low-order digit of the numeral for $x$, times $y$).
Also, the numeral for the number
$\lfloor x \div 10 \rfloor$ is the same as the numeral for $x$, but
without its low-order digit
Therefore, $\lfloor x \div 10 \rfloor \cdot y$ is the number that
comes from multiplying $y$ by the numeral for $x$ without its low-order digit.

In other words, the \{long-division\} equation
provides the basis for computing $xy$.
First, multiply $y$ by the low-order
digit of the numeral for $x$.
Then, multiply $y$ by the number
represented by the other digits of the numeral for $x$.
Finally, shift that product one digit to the left,
and add the number from the first step in the procedure.
It simplifies the procedure a bit to observe that
because the product will be shifted before the addition takes place,
the low-order digit of the sum will be the low-order digit of
the numeral from the first part of the procedure,
when we multiplied $y$ by the low-order digit of the numeral for $x$.

The ``shift'' portion amounts to multiplying by 10: $\lfloor x \div 10 \rfloor \cdot y \cdot 10$
The number from the first step in the procedure is $(x$ mod $10) \cdot y$.
Therefore, the procedure computes
$xy = \lfloor x \div 10 \rfloor \cdot y \cdot 10 + (x$ mod $10) \cdot y$

A digital circuit for multiplication
will use binary numerals rather than decimal numerals.
Except for this detail, the procedure is the same
as with grade-school multiplication of decimal numerals.
In this case, we specialize the \{long division\} equation to the case when $d = 2$.

\hspace{2cm} $xy = \lfloor x \div 2 \rfloor \cdot y \cdot 2 + (x$ mod $2) \cdot y$
\hfill \{long division by 2\} \\

In this case, $(x$ mod $2)$ is the low-order bit in the binary numeral for $x$.
And, $\lfloor x \div 2 \rfloor$ is the number represented by the binary numeral for $x$
without its low-order bit.
That is, if [$x_0$ $x_1$ $x_2$ ...] were the numeral for $x$,
written as a sequence of bits, starting with the low-order bit,
then $x_0$ would be $(x$ mod $2)$ and [$x_1$ $x_2$ ...] would be the numeral for
$\lfloor x \div 2 \rfloor$.

The grade-school procedure, applied to binary numerals, goes as follows:
First, multiply $x_0$ times $y$. The low-order bit of this product will be the low-order bit
of the binary numeral of $xy$. Call it $m_0$. In grade school, this is called ``bringing
down'' the digit. The way is clear because the other part of the product will be shifted.

Then, multiply the other bits of the binary numeral of $x$ times the multiplicand, $y$.
Add the binary numeral for this product to the binary numeral of $x$ without its
low-order bit. Let's give names to the bits of this numeral: [$m_1$ $m_2$ ...]
Then insert the low-order bit from the first product (namely, $m_0$) to form
the numeral for $xy$ = [$m_0$ $m_1$ $m_2$ ...].

There is another aspect of using binary numerals rather than decimal numerals
that simplifies the procedure. It is this:
\begin{enumerate}
  \item If $x_0 = 0$, $x_0 \cdot y = 0$,
        so the empty numeral serves as the numeral for $x_0 \cdot y$.
  \item If $x_0 = 1$,
        $x_0 \cdot y = y$, so the numeral for $y$ serves
        as the numeral for $x_0 \cdot y$.
\end{enumerate}

Therefore, when $x_0$ is zero, the $m_0$ that we ``bring down'' and insert into
[$m_1$ $m_2$ ...] to form the numeral of the product of $x$ and $y$
is, in this case, a zero. On the other hand, when $x_0$ is a one,
the $m_0$ that we bring down is the low-order bit of the numeral $y$.

We are looking for some equations that define a binary multiplication operator.
That is, an operator that delivers the binary numeral for the numbers
represented by its operands, which are also binary numerals.
As usual, we assume that such and operator has already been defined,
and we look for a set of consistent, comprehensive, and computational
equations that it would have to satisfy.

Now, let's focus on the multiplication operator for binary numerals.
Call it ``mul.'' The invocation (mul $x$ $y$),
given binary numerals $x$ and $y$, would
deliver the numeral for the product of numbers that $x$ and $y$ represent.
(We have shifted our nomenclature to reflect our new focus:
$x$ and $y$ now stand for binary numerals
instead of numbers.)
If the numeral $x$ is empty, it denotes the number zero, so the product
will be zero, and the empty numeral will serve as the result.
That gives us one equation:

\hspace{2cm} (mul nil $y$) = nil \hspace{2cm} \hfill \{mul0y\}
\\

If the multiplier $x$ = [$x_0$ $x_1$ $x_2$ ...] is not empty,
then we are going to need to compute the product of the other
high-order bits of $x$ and $y$: $m$ = (mul [$x_1$ $x_2$ ...] $y$).

The low-order bit of $x$ (call it $x_0$, as we did earlier),
must be either zero or one.
If $x_0$ is zero, we bring down a zero and insert it into the product, $m$,
of [$x_1$ $x_2$ ...] and $y$. That gives us another equation:

\hspace{2cm} (mul [0 $x_1$ $x_2$ ...] $y$) = (cons 0 $m$) \hfill \{mul0xy\}
\\

If $x_0$ is one, the computation is a bit more complicated.
To examine it in detail, let $u$, $v$, and $w$ be the numbers
that the numerals $x$, $y$, and $m$ stand for.
That is $u$ = (nmb $x$), $v$ = (nmb $y$), and $w$ = (nmb $m$).
We are trying to compute the product of the numerals $x$ and $y$
That is, we are trying to compute the numeral for $u \cdot v$.

Since $x_0$, the low-order bit of $x$, is one, $u$ is an odd number.
Therefore, $u = 1 + 2 \lfloor u \div 2 \rfloor$,
Also, $w = \lfloor u \div 2 \rfloor \cdot v$
because (nmb [$x_1$ $x_2$ ...]) = $\lfloor u \div 2 \rfloor$.

Therefore,

\hspace{1cm} $u \cdot v$ = $(1 + 2 \lfloor u \div 2 \rfloor) \cdot v$
\hfill $u = 1 + 2 \lfloor u \div 2 \rfloor$

\hspace{1.7cm} = $v + 2 \lfloor u \div 2 \rfloor \cdot v$
\hfill \{distributive law\}

\hspace{1.7cm} = (mod $v$ 2) + $2 \lfloor v \div 2 \rfloor + 2 \lfloor u \div 2 \rfloor \cdot v$ 
\hfill \{long division of $v$ by 2\}

\hspace{1.7cm} = (mod $v$ 2) + $2(\lfloor v \div 2 \rfloor + \lfloor u \div 2 \rfloor \cdot v)$ 
\hfill \{distributive law\}

\hspace{1.7cm} = (mod $v$ 2) + $2(\lfloor v \div 2 \rfloor + w)$
\hfill $w = \lfloor u \div 2 \rfloor \cdot v$

We observe that (mod $v$ 2) is the low-order bit of $y$,
and we recall that (rest $y$) is the numeral for $\lfloor v \div 2 \rfloor$
and that $m$ is the numeral for $w$.
Therefore, we can get the numeral for $u \cdot v$
by inserting low-order bit of $y$, which is (mod $v$ 2),
at the beginning of the sum of the numerals (rest $y$) and $m$.
This amounts to bringing down the low-order bit of $y$
and insert it into the sum of the other bits of $y$ and $m$.

That gives us another equation:

\hspace{2cm} (mul [1 $x_1$ $x_2$ ...] $y$) = (cons (first $y$) (add 0 (rest $y$) $m$)) \hfill \{mul1xy\}
\\

Finally, if $y$ is empty, it denotes zero, so the empty numeral
serves as the product. That gives us one more equation:

\hspace{2cm} (mul $x$ nil) = nil \hfill \{mulx0\}
\\

These equations are comprehensive because $y$ is either empty,
in which case equation \{mx0\} applies,
or $y$ is not empty.
If $y$ is not empty, then either $x$ is empty, in which case
equation \{m0y\} applies, or $x$ is not empty.
If $x$ is not empty, its low-order bit is either zero,
in which case equation \{m0xy\} applies, or it is one,
in which case equation \{m1xy\} applies.

The equations are consistent because in the one case
where they might overlap, namely in case both $x$ and $y$
are empty, they deliver the same result, namely the empty numeral.

They are computational because in the inductive equations,
\{m0xy\} and \{m1xy\}, the first operand in
the invocation of the operator, mul, on the right-hand side
is the numeral [$x_1$ $x_2$ ...], which has fewer bits
than the numeral on the left-hand side, which is
[$x_0$ $x_1$ $x_2$ ...].
So, the inductive invocation is closer to a non-inductive
case than formula on the left-hand side of the equation.

Putting this all together in ACL2 notation leads to the
following definitions.
\label{bignum-mul-defun}
\begin{verbatim}
; (mxy x y) = numeral for (nmb x)*(nmb y)
; ASSUME: y is not nil
(defun mxy (x y) ; x, y: binary numerals
  (if (consp x)
      (let* ((m  (mxy (rest x) y)))
        (if (= (first x) 1)
            (cons (first y) (add 0 (rest y) m)) ; mul1xy
            (cons 0 m)))                        ; mul0xy
      nil))                                     ; mul0y

; (mul x y) = numeral for (* (nmb x) (nmb y))
(defun mul (x y)
  (if (consp y)
      (mxy x y) ; y is not nil, invoke mxy
      nil))                                     ; mulx0
\end{verbatim}

From these definitions, ACL2 successfully finds an
inductive proof of the following theorem.
\label{bignum-mul-thm}
\begin{verbatim}
(defthm bignum-mul-thm
  (= (nmb (mul x y)) (* (nmb x) (nmb y))))
\end{verbatim}

\begin{ExerciseList}
\Exercise
Use an induction on the number of bits in the
multiplier to prove the bignum multiplier theorem (bignum-mul-thm, above).
\end{ExerciseList}


%%% Local Variables:
%%% mode: latex
%%% TeX-master: "book"
%%% End:
