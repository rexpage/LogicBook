\chapter[Computer Systems: Simple Principles Lead to Complex Behavior][Computer Systems]{Computer Systems: Simple Principles Lead to Complex Behavior}

\section{Hardware and Software}

Computer systems, both hardware and software, are some of the
most complicated artifacts that humans have ever created. But at
their very core, computer systems are straightforward
applications of the same logical principles that philosophers
have been developing for more than two thousand years, and
this book will show you how.

The hardware component of a computer system is made up of the visible
pieces of equipment that you are familiar with, such as monitors,
keyboards, printers, web cameras, and USB storage devices.  It also
includes the
components inside the computer, such as chips, cables, hard
drives, DVD drives, and boards.
The properties of hardware devices are largely fixed when the system is
constructed. For example, a hard drive can store 2 gigabytes,
or a cable may be able to transmit 25 different signals
simultaneously---but no more.

The hardware that makes up a computer system is not much different
than the electronics inside a television set or DVD
player. But computer hardware can do something that other
consumer electronic devices cannot; it can respond to
information encoded in a special way, what is
referred to as the software component of the system.  We usually
call the software a ``program.''

You may have heard
about different computer chips and their instruction sets, such
as the Intel chips that power the laptops we are using to
write this book.  Software for these chips consists of a list of
instructions, and that is the way that most general-purpose computer
systems in use today represent software.
But there is nothing magical about this way of thinking about
software; in fact, we believe that thinking about software in this
way obscures its important features.

For example, during the 1980s
and 1990s, special computers were created that used mathematical
functions as the basis for the software.  And other computer
systems have completely different representations for software.
The Internet service Second Life, for example, can be programmed
using Scratch for Second Life (S4SL), and programs in S4SL look
like drawings, not instructions.  An even more graphical view of
software is provided by LabView, which is used by scientists and
engineers all over the world to control laboratory equipment.
If you are at a university, there is a good chance
that LabView is being used at one of the laboratories near you!
And software in LabView looks like a large engineering drawing.


In this book, we present software as a collection of mathematical
equations.  This view is entirely different from, yet consistent with
and equivalent to, the view of
software as a list of instructions or as specialized drawings.
The most important advantage of equations is that they make computer software accessible to anyone who understands high school algebra.

\begin{aside}
Logicians and mathematicians have been studying models of computation
since before computers were invented. This was done, in part, to
answer a deep mathematical question: What parts of mathematics can,
in principle, be fully automated?  In particular, is it possible to
build a machine that can discover all mathematical truths?

As a result, many different models of computation were developed,
including Turing machines, Lambda calculus, partial recursive
functions, unrestricted grammars, Post production rules, random-access
machines, and many others.  Historically, computer science theory has
treated Turing machines as the canonical foundation for computation,
the random-access model more accurately describes modern computers.
The equational model of computation that we use in this book falls
into the Lambda calculus bailiwick.

What is truly remarkable is that all of these different models of
computation are equivalent.  That is,
any computation that can be described in one of the models can also
be described in any of the other models.  An extension of this observation
conjectures that all realizable models
of computation are equivalent.  This conjecture is known as the
Church-Turing Thesis, and it lies at the heart of computer science
theory.

Another remarkable fact is that some problems cannot be solved by a program
written in any of these computational models. Alan
Turing, who many consider to be the first theoretical computer
scientist, was the first to discover such uncomputable problems,
shortly after the logician Kurt G\"odel showed that no formal system
of logic could prove all mathematical truths.  
Incomputability and incompleteness make it mathematically impossible 
to build a machine that can prove all mathematical facts.
Mathematics cannot be fully mechanized, not even in principle.

\caption{Models of Computation}
\label{aside-model-of-computation}
\end{aside}

It is the software that gives a computer system much of its power and
flexibility. An iPhone, for example, has a screen that
can display 441,600 different pixels arranged in a 960x460
grid, but it is the software that determines whether the
iPhone displays an album cover or a weather update.
Software makes the hardware more useful by extending its range of
behavior. For instance, the speakers in an iPhone may be able to produce
only a single tone at a time. But the software can instruct the iPhone to
play a sequence of tones, one after another, that sound like Beethoven's Fifth Symphony.

Think of the hardware as
the parts in a computer that you can see, and the software
as information that tells the computer what to do.
But the distinction between hardware and software is not as
clear cut as this suggests. Many hardware components actually
encode software directly and control other pieces of the system.
In fact, hardware today is designed and built using many techniques
first developed to build large software projects.  And a major theme
of this book is that both hardware and software are realizations of formal
logic.

\section{Structure of a Program}

The distinction between hardware and software is well and good,
but it leaves many questions to the imagination, such as:
\begin{enumerate}
\item How can software affect hardware? Example: Instruct an audio
        device to emit a sound.
\item How can software detect the hardware's status? Example:
        Determine whether a switch is pressed or not.
\item What is the range of instructions that software should be able to give hardware?
        Examples: Add two numbers. Select between two formulas, depending on the results of other computations. Replace one formula by another.
\end{enumerate}

We'll start with the question about the range of instructions.
Each model of computation (see Aside~\ref{aside-model-of-computation})
provides an answer to this question.
There are as many answers as there are
models of computation, and logicians have been very creative when it
comes to constructing such models!  Luckily, these models are
completely equivalent to each other, so we can choose the answer
that is most convenient to us.  And the most convenient answer is
that a program consists of basic mathematical primitives, such as
the the basic operations of arithmetic (addition, multiplication, etc)
and the ability to define new operations
based on previously defined operations.

\begin{aside}
We will use the terms ``operation'' and ``function'' interchangeably.
Some models of computation use these terms to mean different things, but in the model we will use makes it convenient to think of them as the same thing.

So, when we say ``function'' we mean ``operation'' and vice versa. And what we're talking about when we use either term is a transformation that delivers results when supplied with input. We refer to the results delivered by a function (or operation) as its ``value'', and we refer to the input supplied to the function as the parameters of the function. Sometimes we use the term ``argument'' instead of ``parameter'', but we mean the same thing in either case. When talking about the input to an ``operation'', we often use the term ``operand'', but we could also use either of the other terms for input, ``parameter'' or ``argument'', since they all mean the same thing in our model.

To summarize, a function (aka, operation) is supplied with parameters (aka, arguments, operands) and delivers a value.
A formula like $f(x,y)$ denotes the value delivered by the function $f$ when supplied with parameters $x$ and $y$. For example, if $f$ were the arithmetic operation of addition, then $f(x,y)$ would stand for the value $x+y$, $f(2,2)$ would denote 4, and $f(3,7)$ would stand for 10.

\caption{Operations, Functions, Operands, Parameters, and Arguments}
\label{operations-and-functions}
\end{aside}

Once we take the familiar arithmetic functions (that is ``operations'') and the ability to define new functions
as the basic model of computation, we can talk about what it is possible for software to do.
Software can affect
hardware by the values that a function delivers.  For example,
a program---that is, a function---can tell an iPhone what to display
in its screen by delivering a 960x460 matrix of pixels.  Each entry in the matrix
can be a number that represents a color, for example 16,711,680 for
red or 65,280 for green.  Similarly, the hardware can inform the
software of the status of a component by triggering a function defined in the software and supplying the status in the parameters of the function.
For example, the parameters of a particular function could be the coordinates of the pixel selected by touching the screen.  Other gestures, such as tapping or
scrolling, would trigger different operations.

Let's consider a simple example.  We will build a
simple computer device that plays rock-paper-scissors.  The machine
has three buttons, allowing the user to select rock, paper, or
scissors.  The device also has a simple display unit.  After the user
makes a selection, the display unit shows the computer's choice and
lets the user know who won that round.  Our program will make the
selection and determine the winner.  To keep things fair, we will
write this program as two separate functions, so the machine cannot
``see'' the human player's choice.

The first function, called \texttt{emily}\footnote{This function is
named after a child of one of the authors.  The ``real'' Emily
plays the rock-paper-scissors game just like the program developed in this
chapter.}, makes the computer's choice.  This function will deliver
either ``rock'' or ``paper'' or ``scissors'' as its value.
We could use 0, 1, and 2 as shorthand for these values,
but computers are versatile enough
with any type of information, not just numbers, so we are going to stick with the longer names, to make it easier for us to keep track of what things mean.

But what about the input to the function \texttt{emily}?  The value of a mathematical function is completely determined by
its arguments.  That's what it means to be a function! So if
\texttt{emily} has no arguments, it must always return the same value,
and that would be a very boring game.  We've already decided that it
would be unfair for \texttt{emily} to see the player's choice,
but what about the last round?  It is fair for the machine to make its selection by considering the previous round of the game, so the argument can be the user's
\emph{previous} choice---or a special token, like ``N/A'' for the
game's first round.  The function \texttt{emily} can now be described
as follows:
\begin{displaymath}
emily(u) =
   \left\{
        \begin{array}{ll}
        \mbox{``rock''}     & \mbox{if } u = \mbox{``scissors''} \\
        \mbox{``paper''}    & \mbox{if } u = \mbox{``rock''} \\
        \mbox{``scissors''} & \mbox{otherwise}
        \end{array}
   \right.
\end{displaymath}

The second function, which can be called \texttt{score}, decides who
was the winner of the round.  Its input corresponds to the choices
made by the computer and the user, respectively.  Its output consists
of a pair or values.  The first value determines the winner, and the
second ``remembers'' the user's choice.
\begin{displaymath}
score(c,u) =
   \left\{
        \begin{array}{ll}
        (\mbox{``none''}, u)     & \mbox{if } c = u \\
        (\mbox{``computer''}, u) & \mbox{if } (c,u) = (\mbox{``rock''}, \mbox{``scissors''}) \\
        (\mbox{``user''}, u)     & \mbox{if } (c,u) = (\mbox{``rock''}, \mbox{``paper''}) \\
        (\mbox{``computer''}, u) & \mbox{if } (c,u) = (\mbox{``paper''}, \mbox{``rock''}) \\
        (\mbox{``user''}, u)     & \mbox{if } (c,u) = (\mbox{``paper''}, \mbox{``scissors''}) \\
        (\mbox{``computer''}, u) & \mbox{if } (c,u) = (\mbox{``scissors''}, \mbox{``paper''}) \\
        (\mbox{``user''}, u)     & \mbox{if } (c,u) = (\mbox{``scissors''}, \mbox{``rock''})
        \end{array}
   \right.
\end{displaymath}
What is the point of the second value of \texttt{score}?  As you can
see, it is always equal to $u$, i.e., the user's selection.  The
intent is that this second returned value will be passed as the input
to the next call of \texttt{emily}.  This is how the program can
remember the user's previous choice.

We could certainly have dropped this second return value, and simply
stipulated that the hardware is responsible for remembering
the user's last selection.  However, we chose to write the program
in this way for two reasons.  First, it gives us more flexibility.
The hardware is simply responsible for storing the value returned
by \texttt{score} and sending it to \texttt{emily} in the next round.
We can make the program a more sophisticated player of
rock-paper-scissors by changing the software, possibly changing the
value that is passed from \texttt{score} to \texttt{emily} in the
process.  For example, a more sophisticated program may want to
keep track of the number of times that the user has selected each of
the choices.  Since it's the software that decides what to remember
from each round, this choice can be changed very easily.  That is
the flexibility and power of software.

The second reason for writing the program in this way is that it
gives us an opportunity to make an important point about our
computational model.  Since programs in our model consist of collections of equations defining mathematical functions, they cannot ``remember'' anything.
This will come as a surprise to programmers who are
used to other computational models (such as C++ or Java).
In those models, programs store
values in ``variables'' and can use those variables later.
We cannot do that in our purely functional model.  If we did, we
would lose the ability to understand our programs in terms of classical
logic and would have to move into an unfamiliar and
much more complicated domain.

Fortunately, our
model allows the \emph{hardware} to keep track of certain values, in much the
same way that the hardware can tell which button has been pressed.
Our model of computation simply allows the hardware to have some
``hidden'' buttons that can store information and later pass it along to
a function in the form of a parameter.

\section{Deep Blue}

Our model of computation is simple, but it has as much power to specify computations
as any other one. It is reasonable to believe, based on what they
can do, that computers must be much more complicated than that.
But in fact, that's all there is to them.
Power from simplicity, a bargain if there ever was one.

Consider \textit{Deep Blue}, the
computer that beat world champion Gary Kasparov at chess on May 11,
1997.  This program can be written as a function whose input is
an 8x8 matrix of numbers that represent the position of the pieces
on the board after Kasparov's last move.  The function's output is
either the position of the board after its move, or a special
white-flag token used to ``resign'' the game.

In principle, a chess-playing function can be written as follows.
Given the input board, determine all possible legal moves.  If there
are no legal moves, resign.  If there is a legal move that results
in checkmate, do that move.  Otherwise, for each legal move, consider
each of the possible moves by the opponent.  Each one of those results
in a new board, which can then be examined by our chess-playing
function!

It may seem surprising that a function can be defined this
way, but circular definitions of functions are common in mathematics.
We usually refer to them as ``inductive'', but ``circular'' is just as good.
The trick is that the circularity, that is the place in the definition that
refers to the function being defined, represents a reduced level of computation.
Some parts of the definition will not be circular, and any circular reference
will involve parameters that are closer to a non-circular portion of the
definition than the parameters supplied to that portion of the definition.

Functions that have circular (that is, inductive) definitions are sometimes
called ``recursive'' functions. We try to avoid that term because it is
often associated with specialized ways to carry out the computation
that the function describes, but we may slip up from time to time and use the term ``recursive''.

For example, the ``function'' that adds the first five reciprocals of the
natural numbers can be written as
\begin{displaymath}
reciprocals\_5 = 1 + \frac{1}{2} + \frac{1}{3} + \frac{1}{4} +
        \frac{1}{5}
\end{displaymath}
This is obviously not an inductive definition.  For that matter, this
``function'' is really a constant, not a function in the common sense
of the word.  But what about a function that computes the sum of the
first $N$ reciprocals?  That is where we need to use an inductive
definition.  The key idea is that this sum is simply $\frac{1}{N}$
plus the sum of the first $N-1$ reciprocals.  More formally, it is
defined by the following equation
\label{reciprocalsdef}
\begin{displaymath}
reciprocals(n) =
\left\{
        \begin{array}{ll}
                0                               & \mbox{if } n = 0 \\
                reciprocals (n-1) + \frac{1}{n} & \mbox{otherwise}
        \end{array}
\right.
\end{displaymath}

As you can see, the definition of reciprocals is circular, but the parameter
in the circular part is closer to zero than the parameter that arrived in the
first place. And the definition is not circular when the parameter is zero.
So, you can use the definition to compute the value of
        $reciprocals(5)$ by computing $reciprocals(4)$, which is
        computed by first computing $reciprocals(3)$, and so on down
        the line.  Eventually, you come to $reciprocals(0)$, for which
        the definition supplies the value one in non-circular
        fashion.
Then, you can work your way back up the line and arrive that
        conclusion that $reciprocals(5) = \frac{137}{60} \approx 2.3$.

Surely it is surprising that all properties of a function can be
derived from a definition that covers only a few special cases, and
with circularity in the mix to boot. But that is the way it is.
Every computable function has a definition of this form, and
that is the way we will define functions throughout this book.

To get back to the chess-playing function, the problem with our
naive approach is that there are
too many moves to consider.  While the function can, \emph{in principle},
select the best move to play next, \emph{in practice}, the computation
would take too much time. 
The sun would be long dead before the computer could decide on its first move.  
But in fact, \textit{Deep Blue}
did play like this, only it used a massively parallel computer so that
it could consider many moves at the same time, and in most cases it
considered only six to eight moves in the future.  
In practice, \textit{Deep Blue} is a complicated function 
with a large definition.  
But in principle it is just a computer program
like the ones we will study in this book.

\begin{ExerciseList}
\Exercise What happens if you try to compute $reciprocals(-1)$?  How
        about $reciprocals(\frac{1}{2})$?

\Exercise Define a function to compute $factor(k, n)$, which is true
        if $k$ is a factor of $n$ and false otherwise.
        Assume that the formula $mod(k,n)$ returns the remainder
        when dividing $k \div n$.
        For example, $mod(17,5)=2$, $mod(9,2)=1$, and $mod(12,4)=0$

\Exercise Define a function to compute $largest\_factor(n)$,
        the largest factor of $n$ other than $n$ itself.
        For example, $largest\_factor(30)=15$ and
        $largest\_factor(15)=5$.
        Assume that the formula
        $largest\_factor\_upto(n,k)$  returns the largest factor of
        $n$ that doesn't exceed $k$.
        For example, $largest\_factor\_upto(30,7)=6$.
        In your definition, use a formula like $largest\_factor\_upto(n,k)$, 
        with appropriate choices for $n$ and $k$.

\Exercise Define a function to compute $prime(n)$, which is true
        if $n$ is a prime number and false otherwise.
        Refer to the function $largest\_factor$
        from the previous exercise in your definition.

\Exercise Define a function to compute $prime\_reciprocals(n)$,
        the sum of the reciprocals of all the prime numbers that are less than or equal to $n$.
        Your definition will be similar to that of the function $reciprocals$ (page~\pageref{reciprocalsdef}).
        That definition has two formulas, one for the case when $n$ is zero,
        and the other for non-zero values of $n$. 
        The new function will have a third case. 
        It will refer to the function $prime$ from the previous exercise
        to avoid incorporating the reciprocals of numbers that are not primes.
\end{ExerciseList}

        Impress your friends with useless trivia:
        The number $reciprocals(n)$ grows without bound as $n$ becomes large.
        The growth rate is about the same as that of $log(n)$, which isn't very fast.
        The number $prime\_reciprocals(n)$ grows even more slowly, of course,
        but still grows without bound.
        However, the sum of the squares of the reciprocals is bounded.
        Leonhard Euler proved these facts over 200 years ago, during
        a kind of ``Cambrian explosion'' of mathematics initiated a hundred year earlier
        by Isaac Newton and Gottfried Leibniz with their invention of the infinitesimal calculus.

%%% Local Variables:
%%% mode: latex
%%% TeX-master: "book"
%%% End:
